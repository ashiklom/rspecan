\documentclass{beamer}\usepackage[]{graphicx}\usepackage[]{color}
%% maxwidth is the original width if it is less than linewidth
%% otherwise use linewidth (to make sure the graphics do not exceed the margin)
\makeatletter
\def\maxwidth{ %
  \ifdim\Gin@nat@width>\linewidth
    \linewidth
  \else
    \Gin@nat@width
  \fi
}
\makeatother

\definecolor{fgcolor}{rgb}{0.345, 0.345, 0.345}
\newcommand{\hlnum}[1]{\textcolor[rgb]{0.686,0.059,0.569}{#1}}%
\newcommand{\hlstr}[1]{\textcolor[rgb]{0.192,0.494,0.8}{#1}}%
\newcommand{\hlcom}[1]{\textcolor[rgb]{0.678,0.584,0.686}{\textit{#1}}}%
\newcommand{\hlopt}[1]{\textcolor[rgb]{0,0,0}{#1}}%
\newcommand{\hlstd}[1]{\textcolor[rgb]{0.345,0.345,0.345}{#1}}%
\newcommand{\hlkwa}[1]{\textcolor[rgb]{0.161,0.373,0.58}{\textbf{#1}}}%
\newcommand{\hlkwb}[1]{\textcolor[rgb]{0.69,0.353,0.396}{#1}}%
\newcommand{\hlkwc}[1]{\textcolor[rgb]{0.333,0.667,0.333}{#1}}%
\newcommand{\hlkwd}[1]{\textcolor[rgb]{0.737,0.353,0.396}{\textbf{#1}}}%
\let\hlipl\hlkwb

\usepackage{framed}
\makeatletter
\newenvironment{kframe}{%
 \def\at@end@of@kframe{}%
 \ifinner\ifhmode%
  \def\at@end@of@kframe{\end{minipage}}%
  \begin{minipage}{\columnwidth}%
 \fi\fi%
 \def\FrameCommand##1{\hskip\@totalleftmargin \hskip-\fboxsep
 \colorbox{shadecolor}{##1}\hskip-\fboxsep
     % There is no \\@totalrightmargin, so:
     \hskip-\linewidth \hskip-\@totalleftmargin \hskip\columnwidth}%
 \MakeFramed {\advance\hsize-\width
   \@totalleftmargin\z@ \linewidth\hsize
   \@setminipage}}%
 {\par\unskip\endMakeFramed%
 \at@end@of@kframe}
\makeatother

\definecolor{shadecolor}{rgb}{.97, .97, .97}
\definecolor{messagecolor}{rgb}{0, 0, 0}
\definecolor{warningcolor}{rgb}{1, 0, 1}
\definecolor{errorcolor}{rgb}{1, 0, 0}
\newenvironment{knitrout}{}{} % an empty environment to be redefined in TeX

\usepackage{alltt}
\usetheme{Boadilla}
\usecolortheme{seagull}
\usenavigationsymbolstemplate{}

\usepackage{tikz}
\usetikzlibrary{
  positioning,
  calc,
  fit
}

\tikzset{
  majorgrid/.style = {step=1, black, thick},
  minorgrid/.style = {step=0.5, gray, thin},
  tinygrid/.style = {step=0.1, gray!50, very thin},
  pic/.style = {inner sep = 0, outer sep = 0},
  invisible/.style = {opacity = 0},
  shaded/.style = {opacity = 0.5},
  visible/.style = {opacity = 1},
  rtikz/.style = {x = 1pt, y = 1pt},
  ampersand replacement = \&,
  % Selective application of TIKZ styles. E.g.
  % red, onslide = {<3->{blue}}
  onslide/.code args = {<#1>#2}{\only<#1>{\pgfkeysalso{#2}}},
  % Shortcut for hiding on all slides except...
  show/.style = {invisible, onslide = {#1{visible}}}
}

\newcommand{\drawgrid}{%
  \draw[tinygrid] (current bounding box.south west) grid (current bounding box.north east);
  \draw[minorgrid] (current bounding box.south west) grid (current bounding box.north east);
  \draw[majorgrid] (current bounding box.south west) grid (current bounding box.north east);
  \fill[red] (0, 0) circle (1mm);
}

\title[Leaf optical properties]{Leaf optical properties shed light on foliar trait variability at individual to global scales}
\author[\textbf{Shiklomanov}, Serbin, Dietze]{\textbf{Alexey Shiklomanov}, Shawn Serbin, Michael Dietze}
\date{B22C-07}
\IfFileExists{upquote.sty}{\usepackage{upquote}}{}
\begin{document}
  



\section{Introduction - Traits}

\begin{frame}
  \titlepage
\end{frame}

\begin{frame}
  \begin{figure}
    \begin{tikzpicture}
      \node[pic, show = {<2->}] (models) {
        \includegraphics[height=35mm]{\string ~/Pictures/common_figures/modelforest.png}
      };
      \node[below = 2mm of models] (diversity) {\LARGE Functional diversity};
      \node[pic, below = 2mm of diversity, show = {<3->}] (real forest) {
        \includegraphics[height=35mm]{\string ~/Pictures/common_figures/diversity_leaves.png}
      };
    \end{tikzpicture}
  \end{figure}
\end{frame}

%TODO: Highlight LMA
\begin{frame}
  \begin{figure}
    \begin{tikzpicture}
      \node[pic] (diaz figure) {
        \includegraphics[width=\linewidth]{\string ~/Pictures/common_figures/diaz_2014_nature_a.png}
      };
    \end{tikzpicture}
  \end{figure}
\end{frame}





\begin{frame}
  \begin{figure}
    \begin{tikzpicture}[
      hl/.style = {very thick, draw = red, shape = circle, radius = 5mm},
      hl line/.style = {very thick, draw = red, ->},
      lab/.style = {anchor = north, font = {\tiny \itshape}}
      ]
      \node[show = {<1>}] {
\begin{knitrout}
\definecolor{shadecolor}{rgb}{0.969, 0.969, 0.969}\color{fgcolor}
\includegraphics[width=\maxwidth]{figure/sla_base-1} 

\end{knitrout}
      };
      \node[show = {<2>}] {
\begin{knitrout}
\definecolor{shadecolor}{rgb}{0.969, 0.969, 0.969}\color{fgcolor}
\includegraphics[width=\maxwidth]{figure/sla2-1} 

\end{knitrout}
      };
      \node[show = {<3>}] {
\begin{knitrout}
\definecolor{shadecolor}{rgb}{0.969, 0.969, 0.969}\color{fgcolor}
\includegraphics[width=\maxwidth]{figure/sla3-1} 

\end{knitrout}
      };
      \node[show = {<4-6>}] {
\begin{knitrout}
\definecolor{shadecolor}{rgb}{0.969, 0.969, 0.969}\color{fgcolor}
\includegraphics[width=\maxwidth]{figure/sla4-1} 

\end{knitrout}
      };

      \node[show = {<7>}] {
\begin{knitrout}
\definecolor{shadecolor}{rgb}{0.969, 0.969, 0.969}\color{fgcolor}
\includegraphics[width=\maxwidth]{figure/sla5-1} 

\end{knitrout}
      };
      \coordinate (d texana) at (5.15, 2.55);
      \coordinate (f grandifolia) at (-3.15, -1.60);
      \coordinate (d texana loc) at (3, 4.1);
      \coordinate (f grandifolia loc) at (-3.9, 4.1);
      \begin{scope}[show = {<5,6>}]
        \node[hl] (d texana hl) at (d texana) {};
        \node[pic, anchor = north east] (d texana pic) at (d texana loc) {
          \includegraphics[width=20mm]{\string ~/Pictures/common_figures/diospyros-texana-leaves.jpg}
        };
        \node[lab] (d texana lab) at (d texana pic.south) {Diospyros texana};
        \draw[hl line] (d texana hl.west) -- (d texana pic.east);
      \end{scope}
      \begin{scope}[show = {<6>}]
        \node[hl] (f grandifolia hl) at (f grandifolia) {};
        \node[pic, anchor = north west] (f grandifolia pic) at (f grandifolia loc) {
          \includegraphics[width=20mm]{\string ~/Pictures/common_figures/fagus-grandifolia-leaves.jpg}
        };
        \node[lab] (f grandifolia lab) at (f grandifolia pic.south) {Fagus grandifolia};
        \draw[hl line] (f grandifolia hl.north) -- (f grandifolia lab.south);
      \end{scope}
    \end{tikzpicture}
  \end{figure}
\end{frame}



\begin{frame}
  \begin{figure}
    \begin{tikzpicture}[
      map/.style = {anchor = north, outer sep = 1mm}
      ]
      \node[map] (sla map) {
\begin{knitrout}
\definecolor{shadecolor}{rgb}{0.969, 0.969, 0.969}\color{fgcolor}
\includegraphics[width=\maxwidth]{figure/unnamed-chunk-2-1} 

\end{knitrout}
      };
      \node[map, show = {<2->}] (leafn map) at (sla map.south) {
\begin{knitrout}
\definecolor{shadecolor}{rgb}{0.969, 0.969, 0.969}\color{fgcolor}
\includegraphics[width=\maxwidth]{figure/unnamed-chunk-3-1} 

\end{knitrout}
      };
      \node[map, show = {<3->}] (vcmax map) at (leafn map.south) {
\begin{knitrout}
\definecolor{shadecolor}{rgb}{0.969, 0.969, 0.969}\color{fgcolor}
\includegraphics[width=\maxwidth]{figure/unnamed-chunk-4-1} 

\end{knitrout}
      };
      \node[left = of sla map] (sla lab) {Leaf mass per area};
      \node[show = {<2->}] (leafn lab) at (leafn map -| sla lab) {Leaf N content};
      \node[show = {<3->}] (vcmax lab) at (vcmax map -| sla lab) {$V_{c, max}$};
    \end{tikzpicture}
  \end{figure}
\end{frame}

\begin{frame}
  \begin{figure}
    \begin{tikzpicture}[
      quote/.style = {align = left, font = \small \itshape, anchor = north, text width = 70mm}
      ]
      \node[pic] (galadriel) {
        \includegraphics[width=70mm]{\string ~/Pictures/common_figures/Galadriel_at_her_mirror.png}
      };
      \node[quote] (change text) at (galadriel.south) {
        The world has changed. I~feel it in the water. I~feel it in the earth. I~smell it in the air.
      };
      \node[pic, anchor = north west, show = {<3->}] (keeling) at (change text.south) {
        \includegraphics[height=45mm]{\string ~/Pictures/common_figures/keeling_curve.png}
      };
      \node[pic, anchor = east, show = {<2->}] (phenology) at (keeling.west) {
        \includegraphics[height=20mm]{\string ~/Pictures/common_figures/phenology_leaves.jpg}
      };
      \node[pic, below = of keeling, show = {<2>}] {Remote sensing};
    \end{tikzpicture}
  \end{figure}
\end{frame}

\begin{frame}
  \begin{figure}
    \begin{tikzpicture}[
      pic/.style = {outer sep = 0.5pt}
      ]
      \node[pic] (leaf) {
        \includegraphics[height=25mm]{\string ~/Pictures/common_figures/fagus-grandifolia-leaves.jpg}
      };
      \node[pic, anchor = west] (fieldspec) at (leaf.east) {
        \includegraphics[height=25mm]{\string ~/Pictures/common_figures/FieldSpec-4-Wide-Res-Pistol.jpg}
      };
      \node[pic, anchor = north west] (aviris plane) at (leaf.south east) {
        \includegraphics[height=20mm]{\string ~/Pictures/common_figures/aviris_plane.jpg}
      };
      \node[pic, anchor = west] (aviris cube) at (aviris plane.east) {
        \includegraphics[width=20mm, angle = 90]{\string ~/Pictures/common_figures/aviris_image_cube.jpeg}
      };
      \node[pic, anchor = north west, xshift = 3mm] (satellite) at (aviris plane.south east) {
        \includegraphics[height=30mm]{\string ~/Pictures/common_figures/landsat_picture.jpg}
      };
      \node (rs text) at (fieldspec.east -| satellite.north) {\Large Remote sensing};
      \draw[thick, green!70!black, show = {<2>}] (leaf.south west) rectangle (fieldspec.north east);
    \end{tikzpicture}
  \end{figure}
\end{frame}

\section{Introduction - Remote sensing}





\begin{frame}
  \begin{figure}
    \begin{tikzpicture}[
      pic/.style = {outer sep = 1pt},
      sig label/.style = {font = \footnotesize, anchor = south}
      ]
      \node[pic, show = {<1-2>}] (spec plot) {
\begin{knitrout}
\definecolor{shadecolor}{rgb}{0.969, 0.969, 0.969}\color{fgcolor}
\includegraphics[width=\maxwidth]{figure/specintro-1} 

\end{knitrout}
      };
      \node[pic, anchor = north, yshift = -7mm, show = {<2>}] (traits) at (spec plot.south) {
        \includegraphics[height=30mm]{\string ~/Pictures/common_figures/diaz_2014_nature_a.png}
      };
      \draw[<->, thick, show = {<2>}] (spec plot) to node[left]{?} node[right]{?} (traits);
      \node[pic, anchor = north, show = {<3>}] (spectra) at (spec plot.north) {
\begin{knitrout}
\definecolor{shadecolor}{rgb}{0.969, 0.969, 0.969}\color{fgcolor}
\includegraphics[width=\maxwidth]{figure/sigspec2-1} 

\end{knitrout}
      };
      \node[pic, anchor = north, show = {<4>}] (rtm) at (spec plot.north) {
\begin{knitrout}
\definecolor{shadecolor}{rgb}{0.969, 0.969, 0.969}\color{fgcolor}
\includegraphics[width=\maxwidth]{figure/rtm1-1} 

\end{knitrout}
      };
    \end{tikzpicture}
  \end{figure}
\end{frame}





\begin{frame}
  \begin{figure}
    \begin{tikzpicture}[remember picture, overlay]
      \node[anchor = north west, yshift = -3mm, xshift = 1mm] (spectra) at (current page.north west) {Spectra};
      \node[right = of spectra, xshift = 3mm] (traits) {Traits}
      edge [<-, thick] node[align = center] {RTM\\inversion} (spectra);
      \node[draw, thick, shape = rectangle, fit = (spectra) (traits)] {};
    \end{tikzpicture}
    \begin{tikzpicture}
      \node[pic, show = {<2->}] (observed) {
\begin{knitrout}
\definecolor{shadecolor}{rgb}{0.969, 0.969, 0.969}\color{fgcolor}
\includegraphics[width=\maxwidth]{figure/observed_spec-1} 

\end{knitrout}
      };
      \begin{scope}[show = {<3->}]
        \node[pic, anchor = north east, yshift = -5mm, xshift = -10mm] (bad fit) at (observed.south) {
\begin{knitrout}
\definecolor{shadecolor}{rgb}{0.969, 0.969, 0.969}\color{fgcolor}
\includegraphics[width=\maxwidth]{figure/bad_fit-1} 

\end{knitrout}
        };
        \node[pic, left = 12mm of observed] (bad bar) {
\begin{knitrout}
\definecolor{shadecolor}{rgb}{0.969, 0.969, 0.969}\color{fgcolor}
\includegraphics[width=\maxwidth]{figure/bad_bar-1} 

\end{knitrout}
        };
        \draw[->, out = 180, in = 90] (observed.west) to (bad fit.north);
      \end{scope}
      \begin{scope}[show = {<4->}]
        \node[pic, anchor = north west, yshift = -5mm, xshift = 10mm] (good fit) at (observed.south) {
\begin{knitrout}
\definecolor{shadecolor}{rgb}{0.969, 0.969, 0.969}\color{fgcolor}
\includegraphics[width=\maxwidth]{figure/good_fit-1} 

\end{knitrout}
        };
        \node[pic, right = 10mm of observed] (good bar) {
\begin{knitrout}
\definecolor{shadecolor}{rgb}{0.969, 0.969, 0.969}\color{fgcolor}
\includegraphics[width=\maxwidth]{figure/good_bar-1} 

\end{knitrout}
        };
        \draw[->, out = 0, in = 90] (observed.east) to (good fit.north);
      \end{scope}
    \end{tikzpicture}
  \end{figure}
\end{frame}

\begin{frame}{Bayesian RTM inversion}
  \begin{figure}
    \begin{tikzpicture}
      \node[pic] (samples) {
      \includegraphics[width=2in]{{\string ~/Projects/prospect-traits/agu_presentation/inversion.Cab}.png}
      };
      \node[pic, anchor = south west, xshift = -3mm] (bad fit) at (samples.north west) {
\begin{knitrout}
\definecolor{shadecolor}{rgb}{0.969, 0.969, 0.969}\color{fgcolor}
\includegraphics[width=1in,height=1in]{figure/bad_fit-1} 

\end{knitrout}
      };
      \node[pic, anchor = south east, xshift = 3mm] (good fit) at (samples.north east) {
\begin{knitrout}
\definecolor{shadecolor}{rgb}{0.969, 0.969, 0.969}\color{fgcolor}
\includegraphics[width=1in,height=1in]{figure/good_fit-1} 

\end{knitrout}
      };
      \node[pic, anchor = west, yshift = -6mm, xshift = -3mm, show = {<2->}] (normal dist) at (samples.east) {
\begin{knitrout}
\definecolor{shadecolor}{rgb}{0.969, 0.969, 0.969}\color{fgcolor}
\includegraphics[width=\maxwidth,angle=270]{figure/normaldist-1} 

\end{knitrout}
      };
      \node[rotate = 90, anchor = south, font = \small] (chl text) at (samples.west) {Chlorophyll $(\mu g ~ cm^{-2})$};
      \node[anchor = north, font = \small] (iter text) at (samples.south) {Iterations};
      \coordinate (bad start) at (-1.5, 1.8);
      \coordinate (good start) at (1, -0.7);
      \draw[->, thick] (bad start) to (bad fit.south);
      \draw[->, thick] (good start) to (good fit.south);
      \node[anchor = west, xshift = 3mm, show = {<2->}, text width = 35mm] at (normal dist.east) {Trait estimate, with uncertainty};
    \end{tikzpicture}
  \end{figure}
\end{frame}





\begin{frame}
  \begin{figure}
    \begin{tikzpicture}
      \node[pic] (data map) {
\begin{knitrout}
\definecolor{shadecolor}{rgb}{0.969, 0.969, 0.969}\color{fgcolor}
\includegraphics[width=\maxwidth]{figure/data_map-1} 

\end{knitrout}
      };
      \node[pic, anchor = north] (legend) at (data map.south) {
\begin{knitrout}
\definecolor{shadecolor}{rgb}{0.969, 0.969, 0.969}\color{fgcolor}
\includegraphics[width=\maxwidth]{figure/project_legend-1} 

\end{knitrout}
      };
    \end{tikzpicture}
  \end{figure}
\end{frame}

\begin{frame}<1-3>[label=questions]{Research questions}
  \begin{enumerate}
    \item<2-|alert@4> What do we gain by measuring traits from spectra?
    \item<3-|alert@5> How general is this RTM inversion approach?
  \end{enumerate}
\end{frame}

\section{Results}
\againframe<4>{questions}



\begin{frame}{Milkweed water stress experiment \hfill {\tiny J. Couture, unpublished, in ECOSIS}}
  \begin{figure}
    \begin{tikzpicture}
      \begin{scope}[show = {<2->}]
        \node[pic] (traits plot) {
\begin{knitrout}
\definecolor{shadecolor}{rgb}{0.969, 0.969, 0.969}\color{fgcolor}
\includegraphics[width=\maxwidth]{figure/field_trait_plot-1} 

\end{knitrout}
        };
        \node[anchor = south] (traits lab) at (traits plot.north) {Field traits};
      \end{scope}
      \begin{scope}[show = {<3->}]
        \node[pic, right = of traits plot] (spec plot) {
\begin{knitrout}
\definecolor{shadecolor}{rgb}{0.969, 0.969, 0.969}\color{fgcolor}
\includegraphics[width=\maxwidth]{figure/spec_trait_plot-1} 

\end{knitrout}
        };
        \node[anchor = south] (spec lab) at (spec plot.north) {``Optical'' traits};
      \end{scope}
      \coordinate (water box) at (3.8, -1.8);
      \coordinate (chl box) at (6.4, 2);
      \coordinate (car box) at (3.8, 0.1);
      \begin{scope}[very thick, draw = red, show = {<4->}]
        \foreach \point in {water box, chl box, car box}
        \draw (\point) rectangle +(2.4, 1.8);
      \end{scope}
      \node[pic, show = {<5->}, anchor = south] (milkweed spec) at (spec plot.south) {
\begin{knitrout}
\definecolor{shadecolor}{rgb}{0.969, 0.969, 0.969}\color{fgcolor}
\includegraphics[width=\maxwidth]{figure/milkweed_spec_plot-1} 

\end{knitrout}
      };
    \end{tikzpicture}
  \end{figure}
\end{frame}



\begin{frame}{Nitrogen fertilization experiment \hfill {\tiny A.\ Mazur, unpublished, in ECOSIS}}
  \begin{figure}
    \begin{tikzpicture}
      \node[pic] (fert plot) {
\begin{knitrout}
\definecolor{shadecolor}{rgb}{0.969, 0.969, 0.969}\color{fgcolor}
\includegraphics[width=\maxwidth]{figure/fertilization_plot-1} 

\end{knitrout}
      };
      \node[anchor = south] at (fert plot.north) {``Optical'' traits};
      \node[pic, anchor = north west] (fert spec) at (fert plot.north east) {
\begin{knitrout}
\definecolor{shadecolor}{rgb}{0.969, 0.969, 0.969}\color{fgcolor}
\includegraphics[width=\maxwidth]{figure/fertilization_spec-1} 

\end{knitrout}
      };
      \node[pic, anchor = north] (fert legend) at (fert spec.south) {
\begin{knitrout}
\definecolor{shadecolor}{rgb}{0.969, 0.969, 0.969}\color{fgcolor}
\includegraphics[width=\maxwidth]{figure/fertilization_legend-1} 

\end{knitrout}
      };
      \coordinate (car box) at (-2.5, -1.05);
      \coordinate (canth box) at (0, -1.05);
      \coordinate (cab box) at (0, 1.3);
      \coordinate (cm box) at (0, -3.4);
      \begin{scope}[thick, show = {<2->}]
        \foreach \point in {car box, canth box}
        \draw[thick, blue] (\point) rectangle +(2.4, 2.2);
        \foreach \point in {cab box, cm box}
        \draw[thick, red] (\point) rectangle +(2.4, 2.2);
      \end{scope}
    \end{tikzpicture}
  \end{figure}
\end{frame}



\begin{frame}{Needle damage \hfill {\tiny Di Vittorio \& Biging 2009 IJRS}}
  \begin{figure}
    \begin{tikzpicture}
      \node[pic] (needle plot) {
\begin{knitrout}
\definecolor{shadecolor}{rgb}{0.969, 0.969, 0.969}\color{fgcolor}
\includegraphics[width=\maxwidth]{figure/needle_plot-1} 

\end{knitrout}
      };
      \node[pic, anchor = north] (needle legend) at (needle plot.south) {
\begin{knitrout}
\definecolor{shadecolor}{rgb}{0.969, 0.969, 0.969}\color{fgcolor}
\includegraphics[width=\maxwidth]{figure/needle_legend-1} 

\end{knitrout}
      };
    \end{tikzpicture}
  \end{figure}
\end{frame}





\begin{frame}{Phenology \hfill {\tiny Yang et al.\ 2016 RSE}}
  \begin{figure}
    \begin{tikzpicture}
      \node[pic] (cab pheno) {
\begin{knitrout}
\definecolor{shadecolor}{rgb}{0.969, 0.969, 0.969}\color{fgcolor}
\includegraphics[width=\maxwidth]{figure/pheno_cab-1} 

\end{knitrout}
      };
      \node[pic, anchor = west] (car pheno) at (cab pheno.east) {
\begin{knitrout}
\definecolor{shadecolor}{rgb}{0.969, 0.969, 0.969}\color{fgcolor}
\includegraphics[width=\maxwidth]{figure/pheno_car-1} 

\end{knitrout}
      };
      \node[pic, anchor = west] (canth pheno) at (car pheno.east) {
\begin{knitrout}
\definecolor{shadecolor}{rgb}{0.969, 0.969, 0.969}\color{fgcolor}
\includegraphics[width=\maxwidth]{figure/pheno_canth-1} 

\end{knitrout}
      };
    \end{tikzpicture}
  \end{figure}
\end{frame}





\begin{frame}{What drives optical trait variability?}
  \begin{figure}
    \begin{tikzpicture}
      \node[pic] (anova) {
\begin{knitrout}
\definecolor{shadecolor}{rgb}{0.969, 0.969, 0.969}\color{fgcolor}
\includegraphics[width=\maxwidth]{figure/anova_plot-1} 

\end{knitrout}
      };
    \end{tikzpicture}
  \end{figure}
\end{frame}

\againframe<5>{questions}





\begin{frame}
  \begin{figure}
    \begin{tikzpicture}
      \node[pic, show = {<1>}] (chl valid) at (0, 0) {
\begin{knitrout}
\definecolor{shadecolor}{rgb}{0.969, 0.969, 0.969}\color{fgcolor}
\includegraphics[width=\maxwidth]{figure/validation_chl-1} 

\end{knitrout}
      };
      \node[pic, show = {<2>}] (chl r2) at (0, 0) {
\begin{knitrout}
\definecolor{shadecolor}{rgb}{0.969, 0.969, 0.969}\color{fgcolor}
\includegraphics[width=\maxwidth]{figure/r2_chl-1} 

\end{knitrout}
      };
      \node[pic, show = {<3>}] (all r2) at (0, 0) {
\begin{knitrout}
\definecolor{shadecolor}{rgb}{0.969, 0.969, 0.969}\color{fgcolor}
\includegraphics[width=\maxwidth]{figure/r2_all-1} 

\end{knitrout}
      };
      \node[pic, anchor = north] (project legend) at (chl valid.south) {
\begin{knitrout}
\definecolor{shadecolor}{rgb}{0.969, 0.969, 0.969}\color{fgcolor}
\includegraphics[width=\maxwidth]{figure/project_legend-1} 

\end{knitrout}
      };
    \end{tikzpicture}
  \end{figure}
\end{frame}

\begin{frame}[label=summary]{Summary}
  \begin{figure}
    \begin{tikzpicture}
      \matrix{
        \node[show = {<1->}] {Plant functional diversity}; \&
        \node[pic, show = {<1->}] {
          \includegraphics[height=15mm]{\string ~/Pictures/common_figures/diaz_2014_nature_a.png}
        }; \\
        \node[show = {<2->}] {Remote sensing of traits}; \&
        \node[pic, show = {<2->}] {
\begin{knitrout}
\definecolor{shadecolor}{rgb}{0.969, 0.969, 0.969}\color{fgcolor}
\includegraphics[width=\maxwidth,height=15mm,keepaspectratio]{figure/sig_plot-1} 

\end{knitrout}
        }; \\
        \node[show = {<3->}] {Applications of RTM inversion}; \&
        \node[pic, show = {<3->}] {
\begin{knitrout}
\definecolor{shadecolor}{rgb}{0.969, 0.969, 0.969}\color{fgcolor}
\includegraphics[width=\maxwidth,height=15mm,keepaspectratio,clip,trim={0 1.9in 0 0}]{figure/fertilization_plot-1} 

\end{knitrout}
        }; \\
        \node[show = {<4->}] {Pitfalls of RTM inversion}; \&
        \node[pic, show = {<4->}] (pitfalls) {
\begin{knitrout}
\definecolor{shadecolor}{rgb}{0.969, 0.969, 0.969}\color{fgcolor}
\includegraphics[width=\maxwidth,height=20mm,keepaspectratio]{figure/validation_chl-1} 

\end{knitrout}
        };
        \fill[white, visible, onslide = {<4->{invisible}}] (pitfalls.south east) rectangle (pitfalls.north west); \\
      };
    \end{tikzpicture}
  \end{figure}
\end{frame}

\begin{frame}{Acknowledgements}
  \begin{columns}[t]
    \begin{column}{0.5\textwidth}
      \textbf{Co-authors and collaborators}
      \begin{itemize}
        \item Michael Dietze, Boston University
        \item Shawn Serbin, Brookhaven National Lab
        \item Alan Di Vittorio, Berkeley Lab
        \item Adrianna Foster, NASA GSFC
        \item ECOSIS team, U. Wisconsin and U. Minnesota
        \item PEcAn team
      \end{itemize}
    \end{column}
    \begin{column}{0.5\textwidth}
      \textbf{Funding}
      \begin{itemize}
        \item Boston University Department of Earth \& Environment
        \item NASA Grant NNX14AH65G
        \item NASA Earth \& Space Science Fellowship (NNX16AO13H)
      \end{itemize}
    \end{column}
  \end{columns}
\end{frame}

\againframe<4>{summary}

\end{document}
